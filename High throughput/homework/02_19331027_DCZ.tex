\documentclass[a4paper]{article}
\usepackage[a4paper,left=2cm,right=2cm,top=2cm,bottom=2cm]{geometry}
\title{\Huge \bf High throughput\\ \huge The second homework}
\author{Name: Chengzhi Dong \\ Student ID: 19331027}
\date{}

\usepackage{listings}
\usepackage{xcolor}
\usepackage{graphicx}
\usepackage{float}

\usepackage{fancyhdr}
\usepackage{ctex}
\pagestyle{fancy}
\fancyhead[C]{High throughput}
\renewcommand{\headrulewidth}{2pt}
\renewcommand{\footrulewidth}{2pt}

\definecolor{codegreen}{rgb}{0,0.6,0}
\definecolor{codegray}{rgb}{0.5,0.5,0.5}
\definecolor{codepurple}{rgb}{0.58,0,0.82}
\definecolor{backcolour}{rgb}{0.95,0.95,0.92}

\lstdefinestyle{mystyle}{
	backgroundcolor=\color{backcolour},   
	commentstyle=\color{codegreen},
	keywordstyle=\color{magenta},
	numberstyle=\tiny\color{codegray},
	stringstyle=\color{codepurple},
	basicstyle=\ttfamily\footnotesize,
	breakatwhitespace=false,         
	breaklines=true,                 
	captionpos=b,                    
	keepspaces=false,                 
	numbers=left,                    
	numbersep=5pt,                  
	showspaces=false,                
	showstringspaces=false,
	showtabs=false,                  
	tabsize=1
}

\lstset{style=mystyle}

\begin{document}
    \maketitle
    
	\section{Code}
	\subsection{Directory operations}
	\begin{enumerate}
		\item Creat a directory named learning\_python.
		\begin{lstlisting}[language=bash]
mkdir learning_python
		\end{lstlisting}
	
		\item Switch to learning\_python directory
		\begin{lstlisting}[language=bash]
cd learning_python
		\end{lstlisting}		

		\item Put the chrM.fa file in this directory.
		\begin{lstlisting}[language=bash]
wget http://222.200.186.169/highput/data/chrM.fa
		\end{lstlisting}
	
		\item Open vim editor with \textit{vim count.py}
		\begin{lstlisting}[language=bash]
vim count.py
		\end{lstlisting}	
	\end{enumerate}

    \subsection{Python in biology}
Edit your python code in vim editor, then save and exit. The program should include the following:
    \begin{enumerate}
    	\item Calculate the percentage of each base of ATCG in the chrM chromosome.
    	\begin{lstlisting}[language=Python]
#Question 1
def percent_Cal_1(DNA_seq):
    count_A = 0
    count_T = 0 
    count_C = 0
    count_G = 0
    for i in DNA_seq:
        if i == "A":
            count_A += 1
        elif i == "T":
            count_T += 1
        elif i == "C":            
            count_C += 1
        elif i == "G":
            count_G += 1
    all_count = len(DNA_seq)
    precent_A = count_A*100/all_count
    precent_T = count_T*100/all_count
    precent_C = count_C*100/all_count
    precent_G = count_G*100/all_count    

    print("A | Frequency:{0} | Percentage:{1:.2f}%".format(count_A,precent_A))
    print("T | Frequency:{0} | Percentage:{1:.2f}%".format(count_T,precent_T))
    print("C | Frequency:{0} | Percentage:{1:.2f}%".format(count_C,precent_C))
    print("G | Frequency:{0} | Percentage:{1:.2f}%".format(count_G,precent_G))

def percent_Cal_2(DNA_seq):
    count_ls = set(DNA_seq)
    all_count = len(DNA_seq)/100
    for x in count_ls:
        frequency = DNA_seq.count(x)
        percent = frequency/all_count
        print("{0} | Frequency:{1} | Percentage:{2:.2f}%".format(x,frequency,percent))

def percent_Cal_3(DNA_seq):
    count_ls = dict()
    all_count = len(DNA_seq)/100
    for x in DNA_seq:
        if x in count_ls:
            count_ls[x] += 1
        else:
            count_ls[x] = 1
    for key in count_ls:
        count = count_ls[key]
        percent = count/all_count
        print("{0} | Frequency:{1} | Percentage:{2:.2f}%".format(key,count,percent))
    

def text_to_DNA(text_file_name):
    text_file = open(text_file_name)
    lines = text_file.readlines()
    text_file.close()
    line = [x.strip() for x in lines[1:]]
    lines = ''.join(line)
    DNA_seq = lines.upper()
    return DNA_seq

text_file_name = "chrM.fa"
DNA_seq = text_to_DNA(text_file_name)
print("Question1: Calculate the percentage of each base of ATCG in the chrM chromosome by Funtion1.1:")
percent_Cal_1(DNA_seq)
print("Question1: Calculate the percentage of each base of ATCG in the chrM chromosome by by Funtion1.2:")
percent_Cal_2(DNA_seq)
print("Question1: Calculate the percentage of each base of ATCG in the chrM chromosome by by Funtion1.3:")
percent_Cal_3(DNA_seq)

    	\end{lstlisting}
        \item Calculate insulin =  "GIVEQCCTSICSLYQLENYCNFVNQHLCGSHLVEALYLVCGERGFFYTPKT" amino acid frequency in insulin sequence.
        \begin{lstlisting}
#Question 2 
insulin = "GIVEQCCTSICSLYQLENYCNFVNQHLCGSHLVEALYLVCGERGFFYTPKT"
print("Question2: Calculate insulin = GIVEQCCTSICSLYQLENYCNFVNQHLCGSHLVEALYLVCGERGFFYTPKT amino acid frequency in insulin sequence by Funtion1.2:")
percent_Cal_2(insulin)
print("Question2: Calculate insulin = GIVEQCCTSICSLYQLENYCNFVNQHLCGSHLVEALYLVCGERGFFYTPKT amino acid frequency in insulin sequence by Funtion1.3:")
percent_Cal_3(insulin)       
        \end{lstlisting}
        \item Run 5bp windows in the sequence "PRQTEINSEQWENCE" and count the number of occurrences of each window in the sequence.
        \begin{lstlisting}
#Question 3
def percent_Cal_3_1(DNA_seq):
    count_ls = dict()
    for i in range(len(DNA_seq)-4):
        seq_short = DNA_seq[i:i+5]
        if seq_short in count_ls:
            count_ls[seq_short] += 1
        else:
            count_ls[seq_short] = 1
    for key in count_ls:
        count = count_ls[key]
        print("{0} | Count:{1}".format(key,count))

print("Question3: Run 5bp windows in the sequence PRQTEINSEQWENCE and count the number of occurrences of each window in the sequence by Funtion3.1:")
sequence = "PRQTEINSEQWENCE"
percent_Cal_3_1(sequence)      
        \end{lstlisting}
        
        \item Calculate GC percentage in chrM.fa
        \begin{lstlisting}
# Question4
print("Question4: Calculate GC percentage in chrM.fa:")
count_GC = (DNA_seq.count("G")+DNA_seq.count("C"))/len(DNA_seq)
print("GC percentage in chrM: {0:.2f}".format(count_GC))        
        \end{lstlisting}
        
        \item Use \textit{python3 count.py} to run your program from the command line
		\begin{lstlisting}[language=bash]
python3 count.py		
		\end{lstlisting}
	
    \end{enumerate}
    \section{Result}
        \begin{enumerate}
        \item Question1
            \begin{figure}[H]
            \centering
            \includegraphics[scale=1]{Figure/Question1.png}
            \caption{The percentage of each base of ATCG in the chrM chromosome.}
            \end{figure}

        \item Question2
            \begin{figure}[H]
            \centering
            \includegraphics[scale=1]{Figure/Question2.png}
            \caption{Insulin amino acid frequency in insulin sequence.}
            \end{figure}
          
        \item Question3
            \begin{figure}[H]
            \centering
            \includegraphics[scale=1]{Figure/Question3.png}
            \caption{The number of occurrences of each window in the sequence.}
            \end{figure}
                    
        \item Question4
            \begin{figure}[H]
            \centering
            \includegraphics[scale=1]{Figure/Question4.png}
            \caption{GC percentage in chrM.fa.}
            \end{figure}
         
        \end{enumerate}
    
\end{document}